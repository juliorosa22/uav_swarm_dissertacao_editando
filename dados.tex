

% ---
% EXEMPLO PARA CURSO DE POS GRADUAÇÃO
% ---

\instituicao{Instituto Militar de Engenharia}
\programapgdepartamento{Sistemas e Computação}
\nivelestudo{Mestrado} % Graduação, Mestrado ou Doutorado
% % ---
%\titulo{Agentes autônomos para enxame de drones utilizando aprendizado por reforço profundo}
\titulo{Máquinas de Recompensa no Aprendizado por Reforço Multiagente: Um Framework para Controle Escalável de Enxame de VANTs}
\palavraschave{Enxame de VANTs, Aprendizado por Reforço Multiagente, Framework RM-MARL, Máquinas de Recompensa, Controle Descentralizado}
\keywords{UAV Swarm, MARL, RM-MARL Framework, Reward Machines, Decentralized Control}
% % ---
\autores{Júlio César Santana da }{Rosa Filho}% 1+ autores
% % \autor{Fulano de}{Tal} %{nome}{sobrenome}
\orientadores{Paulo Fernando Ferreira}{Rosa}{Ph.D.}%{nomes}{sobrenomes}{títulos}
% % ---
\local{Rio de Janeiro}
\data{2025}
\datadefesa{19 de março de 2025}
% \bancadeexaminadores{
%      Prof. \textbf{Juraci Ferreira Galdino} - D.Sc. do IME - Presidente,
%      Prof. \textbf{Paulo César Pellanda} - Ph.D. do IME,
%      Prof. \textbf{José Adalberto França Junior} - Ph.D. da AGITEC,
%      Prof. \textbf{João José da Cunha e Silva Pinto Ferreira} - Ph.D. da U.Porto
% }

% ---



% ---
% EXEMPLO PARA PROJETO DE FIM DE CURSO
% ---

%\instituicao{Instituto Militar de Engenharia}
%\programapgdepartamento{Engenharia de Fortificação e Construção}
%\nivelestudo{Graduação} % Graduação, Mestrado ou Doutorado
% ---
%\titulo{Modelo Canônico de Trabalho Acadêmico com \abnTeX\space \versaoDocumento}
%\palavraschave{arp,sarp,iot,vant,tarefas cooperativas,agentes inteligentes}
%\keywords{unmanned systems,unmanned vehicles,uav,uas,cooperative tasks,intelligent agents}
% ---
%\autores{Fulano da,Ciclano}{Silva,Pereira}% {nome}{sobrenome} 1+
%\orientadores{Sicrano,Beltrano}{Santos,Oliveira}{Ph.D.,D.Sc.}%{nomes}{sobrenomes}{títulos} 1+
% ---
%\local{Rio de Janeiro}
%\data{2020}
%\datadefesa{30 de fevereiro de 2020}
%\bancadeexaminadores{
%    Prof. \textbf{Orientador 1} - D.Sc. do IME - Presidente,
%    Prof. \textbf{Orientador 2} - D.Sc. do LNCC,
%    Prof. \textbf{Professor 1} - Ph.D. do IMPA,
%    Prof. \textbf{Professor 2} - D.Sc. do LNCC,
%    Prof. \textbf{Professor 3} - D.Sc. do IME,
%    Prof. \textbf{Professor 4} - D.Sc. da PUC
%}

% ---

