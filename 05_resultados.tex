\chapter{Plano de Ação}
\label{metodologia}
\section{Metodologia}
% #TXT_METODOLOGIA
% \textcolor{RedOrange}{Descrição da metodologia (materiais e métodos) a serem utilizadas na pesquisa, ressaltando seus usos para o alcance dos objetivos específicos enumerados no item anterior. Apresentar métodos e técnicas empregadas e com base em que foram escolhidos (justificativas claras). Detalhes sobre os parâmetros escolhidos e os recursos disponíveis também devem ser apresentados. O texto deve demonstrar de modo claro o caminho a ser utilizado para construir a solução proposta.}

%\lipsum[11]
As principais atividades esperadas para o desenvolvimento da proposta são destacadas a seguir:
 \begin{enumerate}
            \item \textbf{Revisão de Literatura}: buscar exemplos de algoritmos MARL (IPPO, QMIX); analisar trabalhos que utilizam máquinas de recompensas (RM) em configurações simples e de múltiplos agentes RL; identificar limitações existentes em projeto de recompensas para múlitplas tarefas.
            \item \textbf{Projetar Máquina de Recompensas}: decompor a missão de rastreamento de alvos em subtarefas gerando os estados da máquina como: desvio de obstáculos, controle de formação, indentificação de alvo.Especificar os eventos que disparam as transições entre os estados (e.g.,SE distancia alvo $<$ 10m ENTÃO mudar para estado de rastreamento).
            
            \item \textbf{Adaptação dos Algoritmos MARL} integrar a máquina de recompensa nos algoritmos MARL para que durante o treinamento do agente seja considerado os estados da máquina $q_{RM}$.\begin{itemize}
                \item Para o algoritmo IPPO: aumentar as observações do agente com o estado da máquina: $o'_i=[o_i,q_{i}]$
                \item Para o algoritmo QMIX: além de extender o espaço de observações, gerar novas experiências sintéticas a partir da técnica CRM (\textbf{Counterfactual Experiences for RM}). Levando em consideração todos os estados da RM para expandir o replay buffer do agente durante o treinamento.
            \end{itemize}

            \item \textbf{Projeto do Ambiente de Simulação}: elaborar os cenários 3D de simulação, modelar os espaços de observação, ações e funções recompensas do ambiente.\begin{itemize}
                \item Utilizar as bibliotecas AirSim e PX4 AutoPilot para modelagem da dinâmica física do drone seus recursos embarcados.
                \item Integrar a modelagem do ambiente com a biblioteca Gymnasium para possibilitar o treinamento dos algoritmos MARL.
                \item Implementar obstaculos e alvos dinâmicos dentro do cenário.
            \end{itemize}
            \item \textbf{Estabelecimento de Métricas}: definir métricas de desempenho para avaliação do treinamento, como: erro de rastreamento (RMSE), taxa de colisões, convergência de treinamento, eficiência energética.

             \item \textbf{Treinamento e Validação}:\begin{itemize}
                 \item Treinar os agentes RM-MARL em cenários progressivos: 1-alvo, obstáculos estáticos (prova de conceito); múltiplos alvos, com obstáculos dinâmicos (teste de escalabilidade).
                 \item Conduzir estudos de hablação: desativar as componentes da máquina de recompensas para avaliar seu impacto na performance;
             \end{itemize}

             \item \textbf{Coleta de Dados e Análise}: realizar analises quantitativa e qualitativa visando melhor compreensão da convergência do treinamento.
             \item \textbf{Iteração e Refinamento}:
             \begin{itemize}
                 \item Refinar o projeto da máquina de recompensas (RM) baseado nos resultados empíricos;
                 \item Otimizar os algoritmos para escalabilidade (e.g., testes com 5, 10 e 20 VANTs).
             \end{itemize}

             \item \textbf{Documentação}: disponibilizar a base de código e o ambiente de simulação de forma open-source, publicação de artigos e escrita da dissertação. 
            %\item Realizar o treinamento dos agentes, testes e simulações utilizando softwares como: ROS , Unreal, AirSim , Gymnasium, e IsaacSim.
            
            %\item Realizar montagem dos drones para início dos testes em campo. 
            
            %\item Fazer o deploy dos agentes treinados para o hardware dos drones, e validar sua efetividade em cenários reais.        
            
            %\item Documentar os aprendizados obtidos.      
 \end{enumerate}
           
        
\section{Resultados Parciais}
\subsection{Implementação}
Realização de simulações e experimentos utilizando algoritmos de aprendizado por reforço profundo como DQN e PPO envolvendo apenas um único agente na execução da tarefa de desvio de obstáculos. Como ilustrado na figura \ref{fig:uav_training_env}.
\begin{figure}[H]
    \centering
    \includegraphics[scale=0.5]{fig/drone_training_photo.png}
    \caption{Captura de tela simulação do treinamento de agente UAV.}
    \fonte{Autor.}
    \label{fig:uav_training_env}
\end{figure}
A figura \ref{fig:uav_swarm_airsim} ilustra os experimentos iniciais no controle de um enxame com 5 drones. Neste cenário o objetivo consiste em treinar o enxame para a navegação em ambiente com obstáculos estáticos mantendo a formação em V.
\begin{figure}[H]
    \centering
    \includegraphics[scale=0.5]{fig/swarm_airsim.png}
    \caption{Simulação Enxame de VANTs com AirSim.}
    \fonte{Autor.}
    \label{fig:uav_swarm_airsim}
\end{figure}
\clearpage

\subsection{Artigos Produzidos}
\begin{itemize}
    \item \textbf{Comparative Analysis of PPO and DQN for UAV Obstacle Avoidance in Simulated Environments}: artigo aceito na 11º Conferência Internacional IFAC que ocorrerá em Julho deste ano na Noruega.
\end{itemize}

\begin{figure}[H]
    \begin{center}
    \includegraphics[scale=0.6]{fig/overview_paths_updated.png}
    \caption{Visualização 2D das trajetórias dos agentes na tarefa desvio de obstáculos. }
    \label{fig:uavs_traj_models}
    \end{center}
\end{figure}

% % Insert this where you want the image to appear
% \begin{figure}[!h]
%     \centering
%     \includegraphics[width=0.8\textwidth]{figuras/HRL_UAV_swarm (1).png}
%     \caption{Hierarquia de subtarefas para o Rastreamento de Alvo pelo enxame.}
%     \label{fig:hrl_hierarchy}
% \end{figure}
%  \newpage

% Para usar tabelas, use sempre o mesmo template abaixo. Altere somente:
% 1. \caption{*} / para colocar o rótulo da tabela em *
% 2. \begin{tabular}{*} / para colocar a formatação da tabela em *
% 3. o conteúdo da tabela (tudo até \end{tabular})


\section{Viabilidade}
\begin{enumerate}
     \item \textbf{Hardware Desktop para simulação}: Processador Intel Core i7 8 x4.2 Ghz 12th Gen, GPU Nvidia RTX 4080 12GB, Memória RAM 64GB. Atualmente o laboratório LIARC conta com computadores de alto desempenho capazes de suportar o volume computacional necessário das simulações.
            
    \item \textbf{Hardware embarcado para drones}: Raspberry PI 5, Nvidia Jetson Nano.
    O LIARC já tem previsto recursos para aquisição no próximo ano dos componentes necessários para construção de 10 drones open hardware, com capacidade de processamento embarcado viabilizando a arquitetura de controle descentralizada.
\end{enumerate}
           


% \begin{table}[h]
% \centering
% \caption{Recursos necessários no desenvolvimento da proposta}
% \vspace{0.5cm}
% \begin{tabular}{c|lr}
 
% Recurso & Descrição \\ % Note a separação de col. e a quebra de linhas
% \hline                               % para uma linha horizontal
% Desktop &  Intel Core i7 12th Gen, 12GB, RAM 64GB. \\
%  &GPU Nvidia RTX 4080
% Hardware embarcado& .938 \\
 
% \label{tab:tabela1}
% \end{tabular}
% \end{table}


\section{Cronograma}
% #TXT_CRONOGRAMA
% \textcolor{RedOrange}{Apresentar o cronograma de execução do projeto considerando como prazo de defesa o seu 24 mês do curso (o texto da Dissertação deverá ser encaminhado à banca, após aprovação da mesma, até o final do 23 mês).}

%\lipsum[11]

O cronograma para o desenvolvimento das atividades relacionadas a esta proposta pode ser visto na figura abaixo.%\ref{fig:cronograma}.

\begin{figure}[!ht]
	\centering
	\includegraphics[scale=0.7]{fig/cronograma_proposta.png}
	\caption{Cronograma da Proposta de Disserta\c{c}\~{a}o.}
	\label{fig:cronograma}
\end{figure}



%\resizebox{\textwidth}{!}{

% \begin{ganttchart}[vgrid,hgrid]{1}{12}  
%     \gantttitle{Cronograma da Pesquisa (Months)}{12} \\  
%     \gantttitlelist{1,...,12}{1} \\  
%     \ganttbar{Revisão de Literatura}{1}{2} \\  
%     \ganttbar{Projeto da RM}{3}{4} \\  
%     \ganttbar{Adaptação dos Algoritmos MARL}{5}{6} \\  
%     \ganttbar{Projeto do Ambiente de Simulação}{3}{5} \\  
%     \ganttbar{Definição de Métricas}{6}{8} \\  
%     \ganttbar{Treinamento e Validação}{9}{10} \\  
%     \ganttbar{Coleta e Análise de Dados}{7}{11} \\
%     \ganttbar{Documentação}{7}{11}
% \end{ganttchart}  

% \begin{figure}[ht]
% \centering
% %\begin{tikzpicture}
%   % Adjust the starting position (x, y) of the Gantt chart
% %\begin{scope}[shift={(0,0)}] % Modify (0,0) to reposition
%     \begin{ganttchart}[
%         vgrid,hgrid,
%         title height=1,
%         bar/.append style={fill=blue!30},
%         bar height=0.6,
%         group/.append style={draw=black, fill=blue!10},
%         milestone/.append style={fill=red, shape=circle, inner sep=2pt},
%         %time slot format=isodate-yearmonth
%       ]{1}{12} % Start and end dates (ISO format)     
%       % Chart title
%       \gantttitle{Research Timeline}{12} \\
%       \gantttitle{2025}{10}
%       \gantttitle{2026}{2} \\
%       \gantttitlelist{1,...,12}{1} \\  
%       %\gantttitlecalendar{year, month} \\
%       % Tasks and milestones
%       \ganttgroup{Etapa Teórica}{1}{3} \\
%       \ganttbar{Revisão de Literatura}{1}{2} \\
%       \ganttbar{Projetar RM VANT}{2}{3} \\
%       \ganttgroup{Implementação}{3}{7}\\
%       \ganttbar{Modificar Algortimos}{3}{5} \\
%       \ganttbar{Modelar Ambiente}{5}{7} \\
%       \ganttmilestone{Proposal Defense}{4} \\
%     \ganttgroup{Etapa Execução}{5}{9}\\
%       \ganttbar{Treinamento dos agentes}{5}{8} \\
%       \ganttbar{Coleta das métricas}{8}{9} \\
%     \ganttgroup{Etapa Análise}{8}{10}\\
%       \ganttbar{Análise das métricas}{8}{9} \\
%       \ganttbar{Refinamento}{9}{10} \\
%     \ganttgroup{Documentação}{10}{12}\\
%       \ganttbar{Escrita Dissertação}{10}{12}
%     \end{ganttchart}
%  %\end{scope}
% %\end{tikzpicture}
% \caption{Research timeline with milestones and dependencies.}
% \label{fig:gantt}
% \end{figure}


%}
% Para usar figuras, use sempre o mesmo template abaixo. altere somente:
% 1. os parâmetros do comando \includegraphics[width=•]{•} / tamanho e arquivo
% 2. \caption{*} / para colocar o rótulo da figura em *
% 3. \label{*} / para colocar a chamada para a figura no texto em *
%    TODA figura deve ter uma chamada no texto e esta deve ser feita sempre no formato:
%    Figura \ref{•} (p. ex. "Figura \ref{fig:cronograma}")

% #CRONOGRAMA
% \begin{figure}[!h]
% 	\centering
% 	\includegraphics[width=0.9\textwidth]{cronograma.eps}
% 	\caption{Cronograma da Proposta de Disserta\c{c}\~{a}o.}
% 	\label{fig:cronograma}
% \end{figure}

% \begin{table}[ht]
%     \centering
%     \resizebox{\textwidth}{!}{
%         \begin{tabular}{|c|c|c|} 
%          \hline
%           Fase & Atividades & Duração  \\ 
%          \hline
%          Revisão de Literatura & Analisar pesquisas na área de MARL e enxame de Drones & OUT 24 à JAN 25  \\ 
%          \hline
%          Projeto dos algoritmos & Desenvolver os algoritmos & JAN 25 à MAR 25  \\
%          \hline
%          Testes em simulação & Implementar algoritmos ambientes simulados & MAR 25 à JUN 25  \\
%          \hline
%         Testes em cenário reais & Fazer deploy dos agentes no hardware dos drones e realizar testes em cenário reais & JUN 25 à SET 25  \\
%          \hline
%          Escrita da dissertação & Produção de artigos e escrita da dissertação & SET 25 à DEZ 25  \\
%          \hline
%          Revisão e correções & realizar ajustes na dissertação & DEZ 25 à JAN 25\\
%      \hline
%     \end{tabular}
% }
% \caption{Cronograma da Proposta de Dissertação}
% \end{table}

