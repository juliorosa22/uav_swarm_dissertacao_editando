\section{Estratégias de Treinamento}
\label{sec:estrategias_treinamento}

Esta seção apresenta as estratégias de treinamento adotadas, incluindo a comparação entre simuladores e as abordagens baseadas em aprendizado curricular com e sem o uso de Reward Machines.

% ---------------------------------------------------------
\subsection{Treinamento no AirSim}
\label{subsec:treinamento_airsim}

% Objetivo exploratório
% Limitações computacionais
% Configuração dos experimentos
% Observações iniciais

% ---------------------------------------------------------
\subsection{Treinamento no IsaacLab}
\label{subsec:treinamento_isaaclab}

% Configuração final de treinamento
% Paralelização
% Estabilidade e escalabilidade
% Vantagens observadas

% ---------------------------------------------------------
\subsection{Comparação de Performance entre Simuladores}
\label{subsec:comparacao_simuladores}

% Comparação qualitativa
% Comparação quantitativa
% Impacto no tempo de treinamento
% Discussão sobre fidelidade física

% ---------------------------------------------------------
\subsection{Abordagem Baseline}
\label{subsec:baseline}

% Curriculum Learning sem Reward Machines
% Definição dos estágios
% Objetivo da baseline

% ---------------------------------------------------------
\subsection{Curriculum Learning com Reward Machines}
\label{subsec:curriculum_rm}

% Integração entre Curriculum Learning e RM
% Definição formal dos estágios:
% 1. Hover
% 2. Point-to-Point
% 3. Desvio de obstáculos (single-agent)
% 4. Navegação em formação em V
% 5. Navegação em enxame com obstáculos
% Discussão da progressão entre estágios
