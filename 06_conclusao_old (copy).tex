\chapter{Conclusão}
A coordenação de enxames de VANTs em ambientes dinâmicos e descentralizados representa um desafio crítico para aplicações militares, logísticas e de resposta a desastres. Este trabalho aborda essa lacuna ao propor a integração de Máquinas de Recompensa (RMs) com algoritmos de Aprendizado por Reforço Multiagente (MARL), estabelecendo um framework inovador para controle descentralizado adaptável a diversas especificações de missão. A principal contribuição reside na capacidade de decompor tarefas complexas—como rastreamento de alvos em movimento e evitação de colisões—em submissões modulares, onde recompensas hierárquicas e contextualizadas orientam os agentes a priorizar ações de forma dinâmica.  

Ao substituir funções de recompensa monolíticas por estruturas baseadas em RMs, o framework RM-MARL resolve ambiguidades na atribuição de crédito, permitindo que VANTs ajustem comportamentos conforme o contexto (e.g., priorizar segurança sobre precisão em zonas de risco). Essa abordagem não apenas melhora a eficiência em missões existentes, como também oferece flexibilidade para adaptação a novos cenários—como rastreamento múltiplo de alvos ou operações em ambientes com interferência de comunicação—sem necessidade de retreinamento completo.  

O resultado esperado é um sistema de controle escalável e interpretável, capaz de gerenciar enxames com dezenas de VANTs em tempo real, alinhando-se a requisitos críticos de setores como defesa e logística. Por exemplo, em operações militares, o framework permitiria que enxames evitassem ameaças dinâmicas enquanto mantêm vigilância contínua, enquanto na logística urbana, garantiria entregas precisas mesmo com obstáculos imprevistos. A combinação de MARL descentralizado com RMs representa um avanço paradigmático, unindo a adaptabilidade de sistemas baseados em dados à precisão de lógica simbólica.  

Trabalhos futuros incluem validação em ambientes reais e extensões para cenários multiobjetivo bem como a transferência de aprendizado para implantação em hardware, consolidando o framework como uma solução versátil para o controle autônomo de enxames de VANTs.  

\paragraph*{\textbf{Agradecimentos}}
Esta pesquisa tem o apoio financeiro parcial do Projeto Enxame de Veículos Autônomos Aéreos e Terrestres: Guiamento, Controle e Navegação [EVAAT-GCN - Convênio 01.24.0661, Ref-3311/24].
agradeco ao meus orientadores, colegas e instituições que contribuíram para o desenvolvimento deste trabalho. A colaboração e o suporte foram fundamentais para alcançar os resultados apresentados. Agradeço também às agências de fomento que possibilitaram a realização desta pesquisa, sem os quais este trabalho não seria possível.

